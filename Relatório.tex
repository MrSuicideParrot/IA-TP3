% !TeX spellcheck = pt_PT
\documentclass[12pt,a4paper]{report}
\usepackage[portuguese]{babel}
\usepackage[utf8]{inputenc} 
\author{André Cirne e José Sousa}
\title{Comparação de Prolog com outras linguagens}

\begin{document}
\maketitle
\tableofcontents
\section{Introdução}
A programação em lógica leva-nos a uma abordagem diferente aos problemas. Por norma numa linguagem imperativa ou orientada ao objeto é necessário pensar passo por passo como resolver, já que em programação lógica, não é necessário ter essa abordagem mas sim conseguirmos definir todos os factos e regras que se relacionam com o problema.

Neste relatório iremos abordar duas linguagens como representantes de paradigmas diferentes, \textit{Prolog} e \textit{Python}. Prolog como linguagem de paradigma lógico e Python com uma linguagem orientada.

A programação en ProLog baseia-se na utilização de perguntas, utilizando encaixe dfs e backtracking  	
\end{document}

