% !TeX spellcheck = pt_PT
\documentclass[12pt,a4paper]{report}
\usepackage[portuguese]{babel}
\usepackage[utf8]{inputenc} 
\author{André Cirne e José Sousa}
\title{Comparação de Prolog com outras linguagens}

\begin{document}
\maketitle
\tableofcontents
\section{Introdução}
A programação em lógica leva-nos a uma abordagem diferente aos problemas. Por norma numa linguagem imperativa ou orientada ao objeto é necessário pensar passo por passo como resolver, já em programação lógica, não é necessário ter essa abordagem mas sim conseguirmos definir todos os factos e regras que se relacionam com o problema.

Neste relatório iremos abordar duas linguagens como representantes de paradigmas diferentes, \textit{Prolog} e \textit{Python}. Prolog como linguagem de paradigma lógico e Python com uma linguagem orientada.

A programação en ProLog baseia-se na utilização de perguntas, que incidem nos factos e regras que foram anteriormente definidos. O Prolog aplica uma prova lógica na tentativa 
\section{Planeamento de viagens}
No problema de planeamento de viagens foi necessário implementar uma resposta a três problemas diferentes. Com base numa base de dados com os voos entre os vários aeroportos, o nosso programa deveria responder aos seguintes problemas: 
\begin{itemize}
 \item Conseguir dizer em que dias é que existem ligações diretas entre dois aeroportos;
 \item Conseguir dizer em que dias é que existem ligações diretas entre dois aeroportos, conseguir dizer quais são os voos que temos de apanhar num determinado dia para conseguir ir de um aeroporto para outro, sendo que não seja necessário existir uma ligação direta e que exista um tempo para transferência entre voos de pelo menos 40 minutos;
 \item Planear uma percurso com inicio e fim na mesma cidade, é necessário que nessa rota passemos por determinados pontos mo entanto, só podemos apanhar um voo por dia, a data de inicio e fim do percurso é dado pelo utilizador.   
\end{itemize}


\subsection{Implementação em Prolog}
\subsection{Implementação em Python}
\section{Analisador sintático para Língua Portuguesa}
\subsection{Implementação em Prolog}
\subsection{Implementação em Python}
\section{Conclusão}
\end{document}

